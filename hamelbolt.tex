\chapter{Hamel Boltzman in astratto}
\begin{itemize}

\item[-] Sia $G$ gruppo di Lie e $x^{i}$ un sistema di coordinate su $G$.
Siano $ \epsilon^{a}$ vettori di una base di 1-forme invarianti a destra.

\item[-]Allora $A^{a}_{j}$ sono le componenti della matrice A delle componenti della base delle 1-forme invarianti a destra sulla base duale $\textrm{d}x^{j}$ tale che
\begin{displaymath}
\left[ \begin{array}{c} \epsilon ^{1} \\ \epsilon ^{2} \\ \vdots \\ \end{array} \right]
 = A \left[ \begin{array}{c} \textrm{d} x^{1} \\ \textrm{d} x^{2} \\ \vdots \\ \end{array} \right] \qquad \epsilon^{a} = A^{a}_{j} \, \textrm{d}x^{j}
\end{displaymath} 

\item[-]di conseguenza
\begin{equation}\label{basedualespuntata}
\textrm{d} x^{j} = B^{j}_{a} \epsilon^{a} \qquad \textrm{con} \qquad A^{i}_{l}B^{l}_{j} = B^{i}_{k} A^{k}_{j} = \delta ^{i}_{j}
\end{equation}
ovvero $A = B^{-1}$ .

Introducendo il duale della base duale, quindi una base di campi invarianti a destra $X_{b}$ risulta che:
\begin{displaymath}
\delta^{a}_{b} = \epsilon^{a}(\,X_{b}) = A^{a}_{j}\, \textrm{d}x^{j}(X_{b})
\end{displaymath}
quindi l'inverso della matrice $A$ è la matrice delle componenti della base dei campi invarianti a destra sulla base naturale associata alle coordinate:
\begin{displaymath}
\Rightarrow \, \textrm{d}x^{j}(X_{b}) = B^{j}_{a} \, \delta^{a}_{b}
\end{displaymath}

\item[-] per valutazione di un arbitrario vettore tangente si ottiene:
\begin{displaymath}
v^{a} = \epsilon^{a} ( v ) = A^{a}_{j} \, \textrm{d}x^{j} (v) = A^{a}_{j} \dot{x}^{j}
\end{displaymath}
che definisce esplicitamente le quasi velocità e la relazione di cambio di coordinate di un generico vettore nelle due basi.

\item[-] Differenziando la formula precedente si ottiene che:

\begin{displaymath}
\textrm{d} v^{a} = A^{a}_{\,j} \, \textrm{d} \dot{x}^{j} + \dfrac{\partial A^{a}_{\,l}}{\partial x^{k}}\, \dot{x}^{l} \, \textrm{d} x^{k}
\end{displaymath}
quindi
\begin{equation}\label{basedualepuntata}
\textrm{d}\dot{x}^{j} = B^{j}_{\,a} \textrm{d} v^{a} - B^{j}_{\,a}\dfrac{\partial A^{a}_{\,l}}{\partial x^{k}}\, \dot{x}^{l} \, \textrm{d} x^{k} = B^{j}_{\,a} \textrm{d} v^{a} - \Bigr(B^{j}_{\,a}  \dfrac{\partial A^{a}_{\,l}}{\partial x^{k}}\, \dot{x}^{l} B^{k}_{\, b} \Bigr) \epsilon^{b}
\end{equation}

le equazioni (\ref{basedualespuntata}) e ( \ref{basedualespuntata}) danno lo sviluppo della base duale sulla base scelta.

\item[-] ora calcola la derivata temporale delle 1-forme invarianti

\begin{displaymath}\begin{split}
\frac{\textrm{d}}{\textrm{d}t} \epsilon^{a} & = \frac{\textrm{d}}{\textrm{d}t}\Bigr( A^{a}_{\: j}\, \textrm{d}x^{j} \Bigr) = \dfrac{\partial A^{a}_{\: j}}{\partial x^{l}}\, \dot{x}^{l}\, \textrm{d}x^{j} + A^{a}_{\: j}\, \textrm{d}\dot{x}^{j} =
\dfrac{\partial A^{a}_{\: j}}{\partial x^{l}}\, \dot{x}^{l}\, B^{j}_{\: b}\, \epsilon^{b} + A^{a}_{\: j} \Bigr( B^{j}_{\: b} \, \textrm{d}v^{b} - B^{j}_{c} \dfrac{\partial A^{c}_{\: l}}{\partial x^{k}}\, \dot{x}^{l} \, B^{k}_{\: b}\Bigr)\, \epsilon^{b} = \\ & =
\textrm{d} v^{a} + \Bigr( \dfrac{\partial A^{a}_{\:j}}{\partial x^{l}}B^{j}_{\: b} -   \dfrac{\partial A^{a}_{\: l}}{\partial x^{j}}B^{j}_{\: b}\Bigr)\, \dot{x}^{l} \, \epsilon^{b}
\end{split}\end{displaymath}


infine elimino $\dot{x}^{b}$ in favore delle quasi velocità ($\dot{x}^{l} = B^{l}_{\,c} v^{c}$). Si ottiene:
\begin{displaymath}
\dfrac{\textrm{d}}{\textrm{d}t} \epsilon^{a} = \textrm{d}v^{a} + \Bigr( \dfrac{ \partial A^{a}_{\,j}}{\partial x^{k}} B^{j}_{\,b}\,B^{k}_{\:c} -
\dfrac{\partial A^{a}_{\,j}}{\partial x^{k}} B^{k}_{\,b}\,B^{j}_{c} \Bigr)\, v^{c} \, \epsilon^{b}
\end{displaymath}

\item[-] si intende dimostrare che il coefficiente tra parentesi coincide con i coefficienti di struttura del gruppo.
I generici campi sulla varietà ammettono la seguente rappresentazione in coordinate sulla base naturale di G: $X = X^{i} \frac{\partial}{\partial x^{i}} $
che chiaramente induce una interpretazione del campo come operatore di derivazione.
Pertanto i campi invarianti a destra risultano:
\begin{equation}\label{eq:componenticampi}
X_{b} = \textrm{d}x^{i}\bigr(X_{b} \bigr)\, \frac{\partial}{\partial x^{i}} = B^{i}_{\, b}\, \frac{\partial}{\partial x^{i}}
\end{equation}

Il commutatore tra i campi invarianti risulta quindi:
\begin{displaymath}\begin{split}
[ X_{b} , X_{c} ] &= \bigr[ B^{j}_{\: b} \frac{\partial}{\partial x^{j}} \,,\, B^{k}_{\: c} \frac{\partial}{\partial x^{k}} \bigr] = B^{j}_{\: b} \frac{\partial}{\partial x^{j}} \bigr( B^{k}_{\: c} \frac{\partial}{\partial x^{k}} \bigr) - B^{k}_{\: c} \frac{\partial}{\partial x^{k}} \bigr( B^{j}_{\: b} \frac{\partial}{\partial x^{j}} \bigr) = \\ &= B^{k}_{\: b} \frac{\partial}{\partial x^{k}} \bigr( B^{j}_{\: c}  \bigr) \frac{\partial}{\partial x^{j}} - B^{k}_{\: c} \frac{\partial}{\partial x^{k}} \bigr( B^{j}_{\: b} \bigr)  \frac{\partial}{\partial x^{j}}  = \bigr( B^{k}_{\: b} \dfrac{\partial B^{j}_{\: c}}{\partial x^{k}} - B^{k}_{\: c} \dfrac{\partial B^{j}_{\: b}}{\partial x^{k}} \bigr) \, \frac{\partial}{\partial x^{j}} = \\ &=
\Bigr( \dfrac{\partial B^{j}_{\: c}}{\partial x^{k}}\, B^{k}_{\: b} \, A^{a}_{\: j} - \dfrac{\partial B^{j}_{\: b}}{\partial x^{k}}\, B^{k}_{\: c} \, A^{a}_{\: j} \Bigr)\, X_{a}  
\end{split}\end{displaymath}
dove sono stati sfruttati nell'ordine il teorema di Scwartz, per cui le derivate seconde si elidono, e l'inverso dell'equazione (\ref{eq:componenticampi}).
Per definizione, visto che in questo caso si parla di campi invarianti a destra, la quantità fra parentesi nell'ultima equazione coincide con le costanti di struttura cambiate di segno:

\begin{displaymath}
-C^{a}_{\,b\,c} = \Bigr( \dfrac{\partial B^{j}_{\: c}}{\partial x^{k}}\, B^{k}_{\: b} \, A^{a}_{\: j} - \dfrac{\partial B^{j}_{\: b}}{\partial x^{k}}\, B^{k}_{\: c} \, A^{a}_{\: j} \Bigr)
\end{displaymath}

Sfruttando il teorema di Liebniz è possibile esprimere l'equazione precedente in un modo equivalente:
\begin{equation}\begin{split}
\label{eq:strutturacampidesta}
-C^{a}_{\,b\,c} &= 
B^{k}_{\: b} \,  \frac{\partial }{\partial x^{k}}\, \bigr( B^{j}_{\: c} \, A^{a}_{\: j} \bigr)- B^{k}_{\: b}\, B^{j}_{\: c} \dfrac{\partial A^{a}_{j}}{\partial x^{k}} - 
B^{k}_{\: c} \,\frac{\partial }{\partial x^{k}}\,\bigr(B^{j}_{\: b} \,  A^{a}_{\: j}\bigr)  
+ B^{k}_{\: c}\, B^{j}_{\: b} \dfrac{\partial A^{a}_{j}}{\partial x^{k}} = \\ &=
B^{k}_{\: c}\, B^{j}_{\: b} \dfrac{\partial A^{a}_{j}}{\partial x^{k}} - B^{k}_{\: b}\, B^{j}_{\: c} \dfrac{\partial A^{a}_{j}}{\partial x^{k}}
\end{split}\end{equation}
in quanto $B^{j}_{\: b}A^{a}_{\: j}= \delta^{a}_{b}$.


\item[-] è possibile ora confrontare l'espressione (\ref{eq:strutturacampidesta}) appena trovata con l'espressione della derivata temporale delle 1-forme invarianti. risulta che

\begin{equation}
\dfrac{\textrm{d}\epsilon^{a}}{\textrm{d}t} = \textrm{d}v^{a} - C^{a}_{\: b \, c}v^{c}\,\epsilon^{b}
\end{equation}
\end{itemize}

Queste forumle sono l'estensione ad un gruppo di Lie arbitrario delle formule di Hamel-boltzman incontrate nella teoria del corpo rigido.
