\documentclass[11pt]{article}

\title{Lie Group theory and application to rigid body mechanics. }
\author{AMM}
\date{}
\usepackage[a4paper,top=4.25cm,bottom=3.75cm,left=3.25cm,right=3cm]{geometry}
\begin{document}
\maketitle

\begin{center}
\begin{tabular}{l c l}
Supervisor & : & Franco Magri \\
\end{tabular}

\end{center}



\begin{abstract}
The purpose of this paper is the presentation of a particular application of the theory of Lie groups in classical mechanics.

Specifically, we will demonstrate how the reduction of the equations of motion of a rigid body with fixed point to the Euler equations is inherent to the structure group owned by the configuration space of the system.

We begin by presenting, in an abstract way, the main features of the structure of Lie group and their realization for a generic matrix set.

Further on, we will present the rotation group in order to emphasize its relationship with the configuration space of rigid body and its possible parameterizations.

Now, we have all the prerequisites to be able to consider the dynamics of rigid bodies specifically. After a brief presentation on the usual derivation of the Euler equations( mainly due to Poisson), we will show their counterpart in the language of group theory in two different ways.

The first method consists in the recognition, within the system of Euler equations, of the components of the coajoint operator of the angular velocity on the angular momentum 1-form.

In our second approach we use the \emph{central equation of dynamics}( introduced by Lagrange) and show how the Euler equations are nothing but the projection of this particular equation on the basis of right invariant 1-forms, objects uniquely related to the Lie group nature of the configuration space of the system.

This particular result has the advantage of being extendable to any conservative mechanical system, on condition that its configuration space is a Lie group. The equations resulting from this process are called \emph{Euler-Poincarè equations}.
\end{abstract}

\end{document}