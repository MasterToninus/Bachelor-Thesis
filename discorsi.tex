\documentclass[11pt]{report}

\title{Parti Scartate}
\author{AMM}


\usepackage{amsmath}
\usepackage{amsfonts}
\usepackage[mathscr]{eucal} 
\usepackage[utf8]{inputenc}
\usepackage[italian]{babel}
\usepackage{listings}
\usepackage{textcomp}
\usepackage{graphicx}
\usepackage{subfigure}
\usepackage{caption}
\usepackage{latexsym}
\usepackage{epstopdf}
\usepackage{eepic,epic,eepicemu}
\usepackage{color}

\usepackage{bm}
\pagestyle{headings}
\definecolor{listinggray}{gray}{0.9}
\definecolor{lbcolor}{rgb}{0.95,0.95,0.95}

\usepackage{amsthm}

\usepackage[a4paper,top=4.25cm,bottom=3.75cm,left=3.25cm,right=3cm]{geometry}

\theoremstyle{plain}
\newtheorem{thm}{Teorema}[section]
\newtheorem{cor}[thm]{Corollario}
\newtheorem{lem}[thm]{Lemma}
\newtheorem{prop}[thm]{Proposizione}

\theoremstyle{definition}
\newtheorem{defn}{Definizione}[chapter]

\theoremstyle{remark}
\newtheorem{oss}{Osservazione}
\newcommand{\HRule}{\rule{\linewidth}{0.5mm}}

\begin{document}
\title
\chapter{discorso sui fluidi}
In un articolo di Arnold (fare riferimento) si mostra come le equazioni di Eulero per il moto di un fluido ideale incompribile possano essere scritte in una forma simile a quelle del corpo rigido con l'opportuna generalizzazione di permettere il fluire continuo del liquido oltre alle rotazioni del liquido.
$\\$
I generali spostamenti, nel caso del corpo rigido, preservano le distanze tra ogni coppia di particelle mentre, per il moto continuo di un liquido, non è possibile imporre alcun vincolo tra le posizioni reciproche delle particelle che lo compongono.
$\\$
Nella sua trattazione Arnold è partito dalla definizione dello spazio di configurazione $Q$ del fluido ideale, ha ottenuto la Lagrangiana sullo spazio $TQ$ e per ultimo è passato al formalismo Hamiltoniano sullo spazio $T^{\ast}Q$.
$\\$
Lo spazio di configurazione di un fluido ideale ha la struttura di gruppo di Lie. Si tratta del gruppo $ G = \textrm{Diff}_{\textrm{Vol}} (D) $ dei diffeomorfismi lisci da un dominio D in se stesso tali da preservare i volumi.
L'operazione di composizione sul gruppo è semplicemente la composizione tra diffeomorfismi.

\section{Dimostrazione diretta delle equazioni di Hamel-Boltman per la rotazione infinitesima}
Chiamo M:
\begin{displaymath}
M : = \left[ \begin{array}{ccc}
\sin(\theta) \sin(\phi) & \cos(\phi)	 & 0  \\
\sin(\theta) \cos(\phi) & -\sin(\phi)	 & 0 \\
\cos(\theta) 			& 0				 & 1 \\
\end{array} \right]
\end{displaymath}
Valgono le seguenti equazioni:
\begin{displaymath}
\left[ \begin{array}{c} \epsilon _{x} \\ \epsilon _{y} \\ \epsilon _{z} \\ \end{array} \right] 
= M 
\left[ \begin{array}{c}
\textrm{d} \psi \\
\textrm{d} \theta \\
\textrm{d} \phi \\
\end{array} \right]
\qquad
\left[ \begin{array}{c}
\textrm{d} \psi \\
\textrm{d} \theta \\
\textrm{d} \phi \\
\end{array} \right] = M^{-1}
\left[ \begin{array}{c}
\epsilon _{x} \\
\epsilon _{y} \\
\epsilon _{z} \\
\end{array} \right] 
\end{displaymath}

\begin{displaymath}
\left[ \begin{array}{c} p \\ q \\ r \\ \end{array} \right] 
= M 
\left[ \begin{array}{c} \dot{\psi} \\ \dot{\theta} \\ \dot{\phi} \\ \end{array} \right]
\qquad
\left[ \begin{array}{c} \dot{\psi} \\ \dot{\theta} \\ \dot{\phi} \\ \end{array} \right] 
= M^{-1}
\left[ \begin{array}{c} p \\ q \\ r \\ \end{array} \right]
\end{displaymath}
Pertanto:

\begin{displaymath}
\dfrac{\textrm{d}}{\textrm{d}t}
\left[ \begin{array}{c} \epsilon _{x} \\ \epsilon _{y} \\ \epsilon _{z} \\ \end{array} \right] 
= M \,
\textrm{d}\Bigr( 
\left[ \begin{array}{c} \dot{\psi} \\ \dot{\theta} \\ \dot{\phi} \\ \end{array} \right] \Bigr)
 + \dfrac{\textrm{d} M}{\textrm{d}t} 
\left[ \begin{array}{c} \textrm{d} \psi \\ \textrm{d} \theta \\ \textrm{d} \phi \\ \end{array} \right] 
= M \Bigr(
\textrm{d}M^{-1} \left[ \begin{array}{c} p \\ q \\ r \\ \end{array} \right] +
M^{-1} \left[ \begin{array}{c} \textrm{d}p \\ \textrm{d}q \\ \textrm{d}r \\ \end{array} \right] \Bigr) + 
\dfrac{\textrm{d}M}{\textrm{d}t} M^{-1} \left[ \begin{array}{c} \epsilon _{x} \\ \epsilon _{y} \\ \epsilon _{z} \\ \end{array} \right] 
\end{displaymath}
sfruttando la regola di Liebniz risulta:

\begin{displaymath}\begin{split}
\dfrac{\textrm{d}}{\textrm{d}t}
\left[ \begin{array}{c} \epsilon _{x} \\ \epsilon _{y} \\ \epsilon _{z} \\ \end{array} \right] 
& = \left[ \begin{array}{c} \textrm{d}p \\ \textrm{d}q \\ \textrm{d}r \\ \end{array} \right] + \Bigr( \textrm{d}( M M^{-1}) - \textrm{d}M \, M^{-1} \Bigr) \left[ \begin{array}{c} p \\ q \\ r \\ \end{array} \right]  + \dfrac{\textrm{d}M}{\textrm{d}t} M^{-1} \left[ \begin{array}{c} \epsilon _{x} \\ \epsilon _{y} \\ \epsilon _{z} \\ \end{array} \right] = \\ &=
\left[ \begin{array}{c} \textrm{d}p \\ \textrm{d}q \\ \textrm{d}r \\ \end{array} \right] + \Bigr(
\dfrac{\textrm{d}M}{\textrm{d}t} M^{-1} \Bigr) \left[ \begin{array}{c} \epsilon _{x} \\ \epsilon _{y} \\ \epsilon _{z} \\ \end{array} \right] -\Bigr(\textrm{d}M \, M^{-1} \Bigr) \left[ \begin{array}{c} p \\ q \\ r \\ \end{array} \right]
\end{split}\end{displaymath}
Sviluppando le derivate si ottiene:

\begin{displaymath}\begin{split}
\dfrac{\textrm{d}M}{\textrm{d}t} M^{-1} &= \Bigr(
\dfrac{\partial M}{\partial \psi}\dot{\psi} + 
\dfrac{\partial M}{\partial \theta}\dot{\theta} +
\dfrac{\partial M}{\partial \phi}\dot{\phi} +
\Bigr) M^{-1} = \left[ \begin{array}{ccc}  \dot{\psi} & \dot{\theta} & \dot{phi} \end{array} \right] \vec{K} M^{-1} = \left[ \begin{array}{ccc}  p & q & r \end{array} 
\right] \bigr( M^{-1} \bigr)^{T} \vec{K} M^{-1} 
\\
\textrm{d}M M^{-1} &=  \left[ \begin{array}{ccc}  \epsilon_{x} & \epsilon_{y} & \epsilon_{z} \end{array} \right] \bigr( M^{-1} \bigr)^{T} \vec{K} M^{-1} 
\end{split}\end{displaymath}
con:

\begin{displaymath} 
\vec{K} =  \left[ \begin{array}{c}  \frac{\partial M}{\partial \psi} \\ \frac{\partial M}{\partial \theta} \\ \frac{\partial M}{\partial \phi} \end{array} \right]  
\qquad \vec{F} = (M^{-1})^{T} \vec{K} M^{-1}
\end{displaymath}
risulta:

\begin{displaymath}
\dfrac{\textrm{d}}{\textrm{d}t}
\left[ \begin{array}{c} \epsilon _{x} \\ \epsilon _{y} \\ \epsilon _{z} \\ \end{array} \right] =
\left[ \begin{array}{c} \textrm{d}p \\ \textrm{d}q \\ \textrm{d}r \\ \end{array} \right] +  \left[ \begin{array}{ccc}  p & q & r \end{array} \right] \vec{F} \left[ \begin{array}{c}  \epsilon_{x} \\ \epsilon_{y} \\ \epsilon_{z} \end{array} \right]
-
\left[ \begin{array}{ccc}  \epsilon_{x} & \epsilon_{y} & \epsilon_{z} \end{array}\right] \vec{F}
\left[ \begin{array}{c}  p \\ q \\ r \end{array}\right]
\end{displaymath}
 
\end{document}




