\documentclass[11pt]{article}

\title{Teoria dei gruppi di Lie e applicazione alla meccanica del corpo rigido.}
\author{AMM}
\date{}

\usepackage{amsmath}
\usepackage{amsfonts}
\usepackage[mathscr]{eucal} 
\usepackage[utf8]{inputenc}
\usepackage[italian]{babel}
\usepackage{listings}
\usepackage{textcomp}
\usepackage{graphicx}
\usepackage{subfigure}
\usepackage{caption}
\usepackage{latexsym}
\usepackage{epstopdf}
\usepackage{eepic,epic,eepicemu}
\usepackage{color}
\usepackage{bm}
\pagestyle{headings}
\definecolor{listinggray}{gray}{0.9}
\definecolor{lbcolor}{rgb}{0.95,0.95,0.95}
\usepackage{amsthm}

\usepackage[a4paper,top=4.25cm,bottom=3.75cm,left=3.25cm,right=3cm]{geometry}




\begin{document}
\maketitle

\begin{center}
\begin{tabular}{l c l}
Relatore & : & Franco Magri \\
Data della Seduta di Laurea & : & 29 novembre 2010 \\
\\
Matricola & : & 072838 \\
CDL & : & Fisica (512) \\
Recapito telefonico & : & 3339825418 \\
\end{tabular}

\end{center}



\begin{abstract}
Lo scopo di questo elaborato è la presentazione di una particolare applicazione della teoria dei gruppi di Lie nell'ambito della meccanica classica.

Nello specifico si intende dimostrare come la riduzione delle equazioni del moto di un corpo rigido con punto fisso alle equazioni di Eulero sia connaturata alla struttura gruppale posseduta dallo spazio di configurazione del sistema.

Si comincia illustrando in modo astratto le principali caratteristiche della struttura di gruppo di Lie e la loro realizzazione per un generico gruppo di matrici.

In seguito viene presentato il gruppo delle rotazioni al fine di sottolineare il suo rapporto con lo spazio di configurazione del corpo rigido e con le sue possibili parametrizzazioni.

A questo punto ci sono tutti i presupposti per poter analizzare nello specifico la dinamica del corpo rigido. Dopo avere presentato brevemente la derivazione usuale, essenzialmente dovuta a Poisson, delle equazioni di Eulero, viene mostrata la loro traduzione nel linguaggio gruppale in due differenti modi.

Il primo modo consiste nel riconoscimento, nel sistema delle equazioni di Eulero, delle componenti dell'azione coaggiunta della velocità angolare sulla 1-forma momento angolare.

Nel secondo modo si fa riferimento all' \emph{equazione centrale della dinamica} di Lagrange e si dimostra come le equazioni di Eulero non siano altro che la proiezione di questa particolare equazione sulla base delle 1-forme invarianti a destra, oggetti legati univocamente alla natura gruppale dello spazio di configurazione del sistema.

Questo particolare risultato ha il pregio di poter essere esteso ad un qualsiasi sistema meccanico conservativo, a condizione che il suo spazio di configurazione sia un gruppo di Lie. Le equazioni che scaturiscono da questo processo sono dette \emph{equazioni di Eulero-Poincarè}.
\end{abstract}

\end{document}




